\documentclass{article}

\usepackage{tabularx}
\usepackage{booktabs}

\title{Problem Statement and Goals \\\progname}

\author{\authname}

\date{}

%% Comments

\usepackage{color}

\newif\ifcomments\commentstrue %displays comments
%\newif\ifcomments\commentsfalse %so that comments do not display

\ifcomments
\newcommand{\authornote}[3]{\textcolor{#1}{[#3 ---#2]}}
\newcommand{\todo}[1]{\textcolor{red}{[TODO: #1]}}
\else
\newcommand{\authornote}[3]{}
\newcommand{\todo}[1]{}
\fi

\newcommand{\wss}[1]{\authornote{blue}{SS}{#1}} 
\newcommand{\plt}[1]{\authornote{magenta}{TPLT}{#1}} %For explanation of the template
\newcommand{\an}[1]{\authornote{cyan}{Author}{#1}}

%% Common Parts

\newcommand{\progname}{ProgName} % PUT YOUR PROGRAM NAME HERE
\newcommand{\authname}{Team \#, Team Name
\\ Student 1 name
\\ Student 2 name
\\ Student 3 name
\\ Student 4 name} % AUTHOR NAMES                  

\usepackage{hyperref}
    \hypersetup{colorlinks=true, linkcolor=blue, citecolor=blue, filecolor=blue,
                urlcolor=blue, unicode=false}
    \urlstyle{same}
                                


\begin{document}

\maketitle

\begin{table}[hp]
\caption{Revision History} \label{TblRevisionHistory}
\begin{tabularx}{\textwidth}{llX}
\toprule
\textbf{Date} & \textbf{Developer(s)} & \textbf{Change}\\
\midrule
09/23/2023 & Nathaniel Hu & Updated project header information, began initial drafting of various sections and subsections \\
09/24/2023 & Ibrahim Issa & Revised Problem Statement, expanded on Environment and Goals subsections \\
09/24/2023 & Mohaansh Pranjal & Added to Stakeholders subsection, revised Environment subsection \\
09/24/2023 & Tushar Aggarwal & Final review of entire document, corrected spelling errors, formatting \\
09/25/2023 & Mohaansh Pranjal & Revised Outputs and Environment subsections \\
09/25/2023 & Nathaniel Hu & Transferred Problem Statement and Goals document text from collaborative document into this LaTeX document; Did some finishing touches \\
03/31/2024 & Ibrahim Issa & Updated and finalized document\\
\bottomrule
\end{tabularx}
\end{table}

\section{Problem Statement}

In this section, the \textbf{problem} that this project's proposed solution aims to solve will be introduced and discussed in more detail.
In particular, the problem will be \textbf{abstracted and characterized} in terms of its \textbf{high-level inputs and outputs}.
The \textbf{stakeholders} of the proposed solution who are impacted by this problem (either \textbf{directly or indirectly}) are also identified and described in more detail.
The \textbf{environment} that this problem exists in, and that the proposed solution is expected to work in, will also be described in further detail here.
Environmental \textbf{constraints} will also be identified and discussed in relation to the problem and potential solution.

\subsection{Problem}

Chest X-rays are the most common medical imaging modality, and they constitute \textbf{40\%} of the \textbf{3.6 billion} medical imaging procedures performed worldwide every year.
Chest X-rays are taken for a wide variety of reasons, \textbf{including} (but not limited to) the following:

\begin{itemize}
\begin{item}
For the discovery of a wide range of \textbf{cardiac and lung conditions}
\end{item}
\begin{item}
To verify the \textbf{position of lines and tubes} in patients in the ICU or during/after interventions
\end{item}
\begin{item}
For ruling out diseases for \textbf{regulatory reasons} such as immigration and occupational health assessments
\end{item}
\end{itemize}

\noindent This huge traffic of data becomes very tedious and time-consuming for \textbf{radiologists} and \textbf{healthcare professionals} to review and analyze.
They need to manually examine each chest X-ray \textbf{carefully} to diagnose a patient’s symptoms or even check for normality.
This arduous process can cause delays for patients who are waiting on their test results.
This is especially problematic for patients in \textbf{time-sensitive situations}, such as those with serious conditions whose symptoms could become worse over time.

\subsection{Inputs and Outputs}

The high-level \textbf{inputs and outputs} of this problem are described as follows:\\
Inputs:
\begin{itemize}
\item Chest X-ray \textbf{image samples}
\item Patient Information
\end{itemize}

\noindent Outputs:
\begin{itemize}
\item Diagnosis Report
\item Disease Prediction
\item Heatmaps
\end{itemize} 

\subsection{Stakeholders}

The various \textbf{stakeholders} of this problem and this project's proposed solution are described in detail below:

\begin{itemize}
\begin{item}
The \textbf{primary stakeholders} for this problem and potential solution are the \textbf{medical professionals} responsible for analyzing medical imaging and performing diagnoses for the patients.
These \textbf{medical professionals} must review many medical images in detail to accurately perform diagnoses, and then document their findings into a \textbf{radiology report}.
\end{item}
\begin{item}
The \textbf{patients} themselves are the \textbf{secondary stakeholders}, as it will be their \textbf{chest X-rays} that will be analyzed.
They are the \textbf{main beneficiaries} of the diagnoses performed on the x-ray images and subsequently produce radiology reports detailing any problems they might have.
These will allow them to get the treatment they need, hopefully within an ideal timeframe.
\end{item}
\begin{item}
Additionally, the \textbf{data/IT departments} of the \textbf{hospitals/medical institutions} using the software are also \textbf{(tertiary) stakeholders}, as \textbf{secure access} to X-ray samples is of importance in such institutions.
The software using medical data is concerned with the department that documents and provides such data.
\end{item}
\end{itemize}

\subsection{Environment}

The \textbf{environment} of this problem and project's proposed solution are described below:

\begin{itemize}
\begin{item}
\noindent \textbf{Software}: The web application would have a \textbf{user-friendly} interface, that can also \textbf{interface with} the hospital’s/medical institution's system using \textbf{secure APIs} as needed.
\end{item}
\begin{item}
\noindent \textbf{Hardware}: The web application would be running on a device that has access to the \textbf{web and the hospital’s/medical institution's server(s)}, as it would need to pull patient's chest X-ray images \textbf{securely}.
\end{item}
\end{itemize}

\section{Goals}

The \textbf{goals} of this project's proposed solution are described in detail below:

\begin{itemize}
\begin{item}
\textbf{Accurate image detection}: Given an X-ray image, the application should \textbf{accurately detect} normality or abnormality with a \textbf{certain range of precision}. To "accurately detect," we aim to minimize the rate of false negatives to increase the ability to detect an abnormality while ensuring the rate of false positives falls within a reasonable range. 
\end{item}
\begin{item}
\textbf{Intuitive user interface}: The user interface should \textbf{facilitate interaction} with the user and should allow the user to \textbf{easily input} an x-ray image and \textbf{output} the diagnostic report.
\end{item}
\begin{item}
\textbf{Diagnostic report generation}: Generate a report \textbf{outlining} any abnormalities in the given X-ray if any. 
\end{item}
\begin{item}
\textbf{Remote access}: The user should be able to \textbf{access} the application through a web browser regardless of location.
\end{item}
\begin{item}
\textbf{Security}: The application should \textbf{protect} user-sensitive user information and be only accessible by \textbf{authorized parties}.
\end{item}
\end{itemize}

\section{Stretch Goals}

The \textbf{stretch goals} of this project's proposed solution are described in detail below:

\begin{itemize}
\begin{item}
\textbf{NLP Report}: Given an X-ray image, generate a \textbf{structured or free-form} radiology report of the diagnostic findings using \textbf{NLP}.
\end{item}
\item \textbf{More Diseases:} Support more disease classifications than the four currently supported.
\end{itemize}

\end{document}