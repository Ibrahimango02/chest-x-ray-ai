\documentclass{article}

\usepackage{float}
\restylefloat{table}

\usepackage{booktabs}

\title{Team Contributions: Rev 0\\\progname}

\author{\authname}

\date{}

%% Comments

\usepackage{color}

\newif\ifcomments\commentstrue %displays comments
%\newif\ifcomments\commentsfalse %so that comments do not display

\ifcomments
\newcommand{\authornote}[3]{\textcolor{#1}{[#3 ---#2]}}
\newcommand{\todo}[1]{\textcolor{red}{[TODO: #1]}}
\else
\newcommand{\authornote}[3]{}
\newcommand{\todo}[1]{}
\fi

\newcommand{\wss}[1]{\authornote{blue}{SS}{#1}} 
\newcommand{\plt}[1]{\authornote{magenta}{TPLT}{#1}} %For explanation of the template
\newcommand{\an}[1]{\authornote{cyan}{Author}{#1}}

%% Common Parts

\newcommand{\progname}{ProgName} % PUT YOUR PROGRAM NAME HERE
\newcommand{\authname}{Team \#, Team Name
\\ Student 1 name
\\ Student 2 name
\\ Student 3 name
\\ Student 4 name} % AUTHOR NAMES                  

\usepackage{hyperref}
    \hypersetup{colorlinks=true, linkcolor=blue, citecolor=blue, filecolor=blue,
                urlcolor=blue, unicode=false}
    \urlstyle{same}
                                


\begin{document}

\maketitle

\section{Demo Plans}

For our project software, we will be demonstrating the key features and
functionalities. The main functionality we will demonstrate is the AI/ML model
used to perform scans of chest X-rays make disease/condition predictions, and
display these results on the web application. \\
We will also be demonstrating the natural language report generation
functionality, performed based on the x-ray scan results.

\section{Meeting Attendance}

The following table shows the number of team meetings attended by each team
member since the POC Demo on November 21, 2023. A total of four team meetings
have been held since then \textit{(according to the number of meeting issues
in the team repo)}.

\begin{table}[H]
  \centering
  \begin{tabular}{ll}
    \toprule
    \textbf{Student} & \textbf{Meetings} \\
    \midrule
    Total & 4 \\
    Allison Cook & 3 \\
    Ibrahim Issa & 2 \\
    Mohaansh Pranjal & 4 \\
    Nathaniel Hu & 2 \\
    Tushar Aggarwal & 2 \\
    \bottomrule
  \end{tabular}
\end{table}

\noindent \textit{Tushar Aggarwal was out of the country for the duration of
the break till the start of the second week of classes.}

\section{Lecture Attendance}

The following table shows the number of lectures attended by each team member
since the POC Demo on November 21, 2023. A total of two lectures have been held
since then \textit{(according to the Google Calendar and the number of lecture
issues in the team repo)}.

\begin{table}[H]
  \centering
  \begin{tabular}{ll}
    \toprule
    \textbf{Student} & \textbf{Lectures} \\
    \midrule
    Total & 2 \\
    Allison Cook & 2 \\
    Ibrahim Issa & 1 \\
    Mohaansh Pranjal & 1 \\
    Nathaniel Hu & 2 \\
    Tushar Aggarwal & 1 \\
    \bottomrule
  \end{tabular}
\end{table}

\section{Commits}

The following table shows the number of commits to the main branch that have
been made by each team member since the POC Demo on November 21, 2023. A total
of 16 commits were made by the entire team since then (according to the team
repository's commit history). The proportion of commits made by each team member
is also shown below (out of a total of 100\%). \\
\textit{(This table does not include the commit count for this submission.)}

\begin{table}[H]
  \centering
  \begin{tabular}{lll}
    \toprule
    \textbf{Student} & \textbf{Commits} & \textbf{Percent} \\
    \midrule
    Total & 16 & 100\% \\
    Allison Cook & 2 & 12.5\% \\
    Ibrahim Issa & 2 & 12.5\% \\
    Mohaansh Pranjal & 5 & 31.3\% \\
    Nathaniel Hu & 2 & 12.5\% \\
    Tushar Aggarwal & 5 & 31.3\% \\
    \bottomrule
  \end{tabular}
\end{table}

\noindent \textit{The team member Nathaniel Hu has at least 3 commits on an
unmerged branch (backend/feature/database-and-user-auth-mgmt) related to changes
for connecting the application's frontend web pages to the backend database. This
branch also involves the backend code for user authentication, patient data
management and interfacing the backend with external IT systems (i.e. of
participating medical institutions). As more changes are made, the number of
commits will very likely go up as well by the time the branch is merged into the
main branch.}

\section{Issue Tracker}

The following table shows the number of issues authored and assigned to each team
member. These numbers include open and closed authored issues, and closed assigned
issues.

\begin{table}[H]
  \centering
  \begin{tabular}{lll}
    \toprule
    \textbf{Student} & \textbf{Authored (O+C)} & \textbf{Assigned (C only)} \\
    \midrule
    Allison Cook & 10 & 9 \\
    Ibrahim Issa & 9 & 7 \\
    Mohaansh Pranjal & 15 & 13 \\
    Nathaniel Hu & 18 & 12 \\
    Tushar Aggarwal & 56 & 55 \\
    \bottomrule
  \end{tabular}
\end{table}

\section{CI/CD}

In our project, CI/CD is currently used to manage the following workflows
(GitHub Actions):
\begin{enumerate}
  \item Create a branch from issue
  \item Autolink the issue to the Pull Request on merge
  \item Perform code checks: Run Linters and check formats and run some unit tests
  \item Perform PR checks: If the template for the Pull Request is filled in
    properly such as adding labels, adequate comments are added and at least one
    issue is linked to a pull request.
\end{enumerate}
The CI/CD will be further expanded upon and used to run more unit tests for PRs,
automatically apply labels to PRs based upon code changes (e.g. frontend vs.
backend vs. documentation code changes) and perform other pre-PR code checks.

\end{document}
