\documentclass[12pt, titlepage]{article}

\usepackage{amsmath, mathtools}

\usepackage[round]{natbib}
\usepackage{amsfonts}
\usepackage{amssymb}
\usepackage{graphicx}
\usepackage{colortbl}
\usepackage{xr}
\usepackage{hyperref}
\usepackage{longtable}
\usepackage{xfrac}
\usepackage{tabularx}
\usepackage{float}
\usepackage{siunitx}
\usepackage{booktabs}
\usepackage{multirow}
\usepackage[section]{placeins}
\usepackage{caption}
\usepackage{fullpage}

\hypersetup{
bookmarks=true,     % show bookmarks bar?
colorlinks=true,       % false: boxed links; true: colored links
linkcolor=red,          % color of internal links (change box color with linkbordercolor)
citecolor=blue,      % color of links to bibliography
filecolor=magenta,  % color of file links
urlcolor=cyan          % color of external links
}

\usepackage{array}

\externaldocument{../../SRS/SRS}

%% Comments

\usepackage{color}

\newif\ifcomments\commentstrue %displays comments
%\newif\ifcomments\commentsfalse %so that comments do not display

\ifcomments
\newcommand{\authornote}[3]{\textcolor{#1}{[#3 ---#2]}}
\newcommand{\todo}[1]{\textcolor{red}{[TODO: #1]}}
\else
\newcommand{\authornote}[3]{}
\newcommand{\todo}[1]{}
\fi

\newcommand{\wss}[1]{\authornote{blue}{SS}{#1}} 
\newcommand{\plt}[1]{\authornote{magenta}{TPLT}{#1}} %For explanation of the template
\newcommand{\an}[1]{\authornote{cyan}{Author}{#1}}

%% Common Parts

\newcommand{\progname}{ProgName} % PUT YOUR PROGRAM NAME HERE
\newcommand{\authname}{Team \#, Team Name
\\ Student 1 name
\\ Student 2 name
\\ Student 3 name
\\ Student 4 name} % AUTHOR NAMES                  

\usepackage{hyperref}
    \hypersetup{colorlinks=true, linkcolor=blue, citecolor=blue, filecolor=blue,
                urlcolor=blue, unicode=false}
    \urlstyle{same}
                                


\begin{document}

\title{Module Interface Specification for \progname{}}

\author{\authname}

\date{\today}

\maketitle

\pagenumbering{roman}

\section{Revision History}

\begin{tabularx}{\textwidth}{p{3cm}p{2cm}X}
\toprule {\bf Date} & {\bf Version} & {\bf Notes}\\
\midrule
01/11/2024 & 0.0 & Initial Document \\
01/12/2024 & 0.1 & Started adding MIS for Module sections \\
01/14/2024 & 0.2 & Added in Table from Module Guide (MG) into Module
  Decomposition section \\
01/16/2024 & 0.3 & Made small updates to Table 1 in Module Decomposition
  section; Continued adding MIS for Module sections \\
01/17/2024 & 0.4 & Completed the Symbols, Abbreviations and Acronyms and
  Introduction sections; Finished adding in MIS for Module sections \\
\bottomrule
\end{tabularx}

~\newpage

\section{Symbols, Abbreviations and Acronyms}

This section records the symbols, abbreviations and acronyms information for
easy reference for terms used in this document. \\

For information on most of the symbols, abbreviations and acronyms referenced
in this document, see the SRS Documentation at the following link:
\url{https://github.com/tusharagg1/chest-x-ray-ai/blob/main/docs/SRS/SRS.pdf}. \\

The information on the rest of the symbols, abbreviations and acronyms
referenced in this document are shown in the table below. \\

\renewcommand{\arraystretch}{1.2}
\begin{tabular}{l l} 
  \toprule    
  \textbf{symbol} & \textbf{description} \\
  \midrule 
  AI/ML & Artificial Intelligence/Machine Learning \\
  \multirow{3}{*}{DICOM} & Digital Imaging and Communications in Medicine; \\
  & technical standard for digital storage/transmission \\
  & of medical images and related information \\
  GUI & Graphical User Interface \\
  \multirow{2}{*}{JPEG/JPG} & Joint Photographic Experts Group; digital image \\
  & compression standard, image format \\
  M & Module \\
  MG & Module Guide \\
  MVC & Model-View-Controller Software Architecture \\
  NLP & Natural Language Processing \\
  SRS & Software Requirements Specification \\
  \multirow{3}{*}{\progname} & The Process of Designing and Developing Software; \\
  & a reference to the software application described \\
  & in this document \\
  \bottomrule
\end{tabular} \\

\newpage

\tableofcontents

\newpage

\pagenumbering{arabic}

\section{Introduction}

The following document details the Module Interface Specifications for
the [Your Program Name Here] software application. This software application
(sometimes referred to as Software Engineering in this document) performs
scans of chest x-ray images, looking for diseases/infections and making
predictions. Those scan results and predictions of diseases/infections are
then put into natural language radiology reports (or components) and returned. \\

Complementary documents include the System Requirement Specifications and
Module Guide. The full documentation and implementation can be found at
\url{https://github.com/tusharagg1/chest-x-ray-ai/tree/main}.

\section{Notation}

The structure of the MIS for modules comes from \citet{HoffmanAndStrooper1995},
with the addition that template modules have been adapted from
\cite{GhezziEtAl2003}.  The mathematical notation comes from Chapter 3 of
\citet{HoffmanAndStrooper1995}.  For instance, the symbol := is used for a
multiple assignment statement and conditional rules follow the form $(c_1
\Rightarrow r_1 | c_2 \Rightarrow r_2 | ... | c_n \Rightarrow r_n )$.

The following table summarizes the primitive data types used by \progname. 

\begin{center}
\renewcommand{\arraystretch}{1.2}
\noindent 
\begin{tabular}{l l p{7.5cm}} 
\toprule 
\textbf{Data Type} & \textbf{Notation} & \textbf{Description}\\ 
\midrule
character & char & a single symbol or digit\\
integer & $\mathbb{Z}$ & a number without a fractional component in (-$\infty$, $\infty$) \\
natural number & $\mathbb{N}$ & a number without a fractional component in [1, $\infty$) \\
real & $\mathbb{R}$ & any number in (-$\infty$, $\infty$)\\
\bottomrule
\end{tabular} 
\end{center}

\noindent
The specification of \progname \ uses some derived data types: sequences, strings, and
tuples. Sequences are lists filled with elements of the same data type. Strings
are sequences of characters. Tuples contain a list of values, potentially of
different types. In addition, \progname \ uses functions, which
are defined by the data types of their inputs and outputs. Local functions are
described by giving their type signature followed by their specification.

\section{Module Decomposition}

The following table is taken directly from the Module Guide document for this project.

\begin{table}[H]
  \centering
  \begin{tabular}{p{0.3\textwidth} p{0.6\textwidth}}
    \toprule
    \textbf{Level 1} & \textbf{Level 2} \\
    \midrule

    {Hardware-Hiding Module} & MedInstInter \\
    \midrule

    \multirow{8}{0.3\textwidth}{Behaviour-Hiding Module} & ChestXRayRead \\
    & ResultsGen \\
    & RepCompGen \\
    & DatabaseOps \\
    & UserAuthMgmt \\
    & Login \\ 
    & PerfScan \\
    & ViewResults \\
    \midrule

    \multirow{5}{0.3\textwidth}{Software Decision Module} & AIModel \\
    & NLPModel \\
    & Backend \\
    & AppController \\
    & AppGUI \\
    \bottomrule

  \end{tabular}
  \caption{Module Hierarchy}
  \label{TblMH}
\end{table}

~\newpage

\section{MIS of Medical Institution Interface Module} \label{mMedInstInter}

\subsection{Module}
MedInstInter

\subsection{Uses}
N/A

\subsection{Syntax}

\subsubsection{Exported Constants}
N/A

\subsubsection{Exported Access Programs}

\begin{center}
  \begin{tabular}{p{3cm} p{4cm} p{4cm} p{3cm}}
    \hline
    \textbf{Name} & \textbf{In} & \textbf{Out} & \textbf{Exceptions} \\
    \hline
    connectToInst & instID: str, credentials: str & connectionStatus: bool &
      InvalidCredentialsException, InstNotFoundException \\
    \hline
  \end{tabular}
\end{center}

\subsection{Semantics}

\subsubsection{State Variables}

\begin{itemize}
  \item connectedInsts: Set(str) - maintains a set of connected institution IDs.
\end{itemize}

\subsubsection{Environment Variables}

\begin{itemize}
  \item InstsITSys: Set(str) - the set of external IT systems the application
    interfaces with to retrieve/exchange information.
\end{itemize}

\subsubsection{Assumptions}
\begin{itemize}
  \item Patient data is stored in the medical institution's database, and the
    software intends to interface with their server to access that information.
\end{itemize}

\subsubsection{Access Routine Semantics}

\noindent connectToInst():
\begin{itemize}
  \begin{item}
    transition:
    \begin{itemize}
      \item Adds `instID' to `connectedInsts' if the provided `credentials' is
        valid.
    \end{itemize}
  \end{item}
  \begin{item}
    output:
    \begin{itemize}
      \item `connectionStatus' is set to True if the connection is successful,
        False otherwise.
    \end{itemize}
  \end{item}
  \begin{item}
    exception:
    \begin{itemize}
      \item Throws `InvalidCredentialsException' if the provided credentials are invalid.
      \item Throws `InstNotFoundException' if the specified 'instID' does not exist.
    \end{itemize}
  \end{item}
\end{itemize}

\subsubsection{Local Functions}
N/A

\newpage

\section{MIS of Chest X-Ray Read Module} \label{mChXRR}

\subsection{Module}
ChestXRayRead

\subsection{Uses}
N/A

\subsection{Syntax}

\subsubsection{Exported Constants}
N/A

\subsubsection{Exported Access Programs}

\begin{center}
  \begin{tabular}{p{3cm} p{4cm} p{4cm} p{3cm}}
    \hline
    \textbf{Name} & \textbf{In} & \textbf{Out} & \textbf{Exceptions} \\
    \hline
    processImg & img: JPEG image & procImg: Processed Image &
      InvalidImageFormatException \\
    \hline
  \end{tabular}
\end{center}

\subsection{Semantics}

\subsubsection{State Variables}
N/A

\subsubsection{Environment Variables}
N/A

\subsubsection{Assumptions}
N/A

\subsubsection{Access Routine Semantics}

\noindent processImg():
\begin{itemize}
  \begin{item}
    transition:
    \begin{itemize}
      \item Initiates the AIModel module to process the provided `img'.
    \end{itemize}
  \end{item}
  \begin{item}
    output:
    \begin{itemize}
      \item `procImg' contains the processed information and findings from the
        chest X-ray analysis.
    \end{itemize}
  \end{item}
  \begin{item}
    exception:
    \begin{itemize}
      \item Throws `InvalidImageFormatException' if the provided `image' is
        not a JPEG or JPG image. 
    \end{itemize}
  \end{item}
\end{itemize}

\subsubsection{Local Functions}
N/A

\newpage

\section{MIS of Results Generation Module} \label{mResGen}

\subsection{Module}
ResultsGen

\subsection{Uses}
N/A

\subsection{Syntax}

\subsubsection{Exported Constants}
N/A

\subsubsection{Exported Access Programs}

\begin{center}
  \begin{tabular}{p{3cm} p{4cm} p{4cm} p{3cm}}
    \hline
    \textbf{Name} & \textbf{In} & \textbf{Out} & \textbf{Exceptions} \\
    \hline
    generateResults & procImg: Processed Image & classification: Disease
      Classification & - \\
    \hline
  \end{tabular}
\end{center}

\subsection{Semantics}

\subsubsection{State Variables}
N/A

\subsubsection{Environment Variables}
N/A

\subsubsection{Assumptions}
N/A

\subsubsection{Access Routine Semantics}

\noindent generateResults():
\begin{itemize}
  \begin{item}
    transition:
    \begin{itemize}
      \item Utilizes the AIModel module to interpret the processed image and
        generate a disease classification. 
    \end{itemize}
  \end{item}
  \begin{item}
    output:
    \begin{itemize}
      \item `classification' contains the generated disease classification for
        each disease.
    \end{itemize}
  \end{item}
  \begin{item}
    exception: N/A 
  \end{item}
\end{itemize}

\subsubsection{Local Functions}
N/A

\newpage

\section{MIS of Report Component Generation Module} \label{mRepCompGen}

\subsection{Module}
RepCompGen

\subsection{Uses}
N/A

\subsection{Syntax}

\subsubsection{Exported Constants}
N/A

\subsubsection{Exported Access Programs}

\begin{center}
    \begin{tabular}{p{3cm} p{3.5cm} p{5cm} p{3cm}}
      \hline
      \textbf{Name} & \textbf{In} & \textbf{Out} & \textbf{Exceptions} \\
      \hline
      generateReport & diagnosis: Disease Diagnosis & report: Radiology Report
        & - \\
      \hline
    \end{tabular}
\end{center}

\subsection{Semantics}

\subsubsection{State Variables}
N/A

\subsubsection{Environment Variables}
N/A

\subsubsection{Assumptions}

\subsubsection{Access Routine Semantics}

\noindent generateReport():
\begin{itemize}
  \begin{item}
    transition: N/A
  \end{item}
  \begin{item}
    output:
    \begin{itemize}
      \item `report' contains the generated radiology report based on the
        provided disease diagnosis.
    \end{itemize}
  \end{item}
  \begin{item}
    exception: N/A
  \end{item}
\end{itemize}

\subsubsection{Local Functions}
N/A

\newpage

\section{MIS of Database Operations Module} \label{mDatabaseOps}

\subsection{Module}
DatabaseOps

\subsection{Uses}
N/A

\subsection{Syntax}

\subsubsection{Exported Constants}
N/A

\subsubsection{Exported Access Programs}

\begin{center}
  \begin{tabular}{p{3cm} p{5cm} p{4cm} p{5cm}}
    \hline
    \textbf{Name} & \textbf{In} & \textbf{Out} & \textbf{Exceptions} \\
    \hline
    storeReport & report: Radiology Report, patientID: str & success: bool &
      ReportStorageException \\
    retrieveReport & patientID: str & report: Radiology Report &
      ReportRetrievalException \\
    \hline
  \end{tabular}
\end{center}

\subsection{Semantics}

\subsubsection{State Variables}
\begin{itemize}
    \item `connectDatabase: bool' indicates whether the module is currently connected to the database.
\end{itemize}

\subsubsection{Environment Variables}
N/A

\subsubsection{Assumptions}
N/A

\subsubsection{Access Routine Semantics}

\noindent storeReport():
\begin{itemize}
  \begin{item}
    transition:
    \begin{itemize}
      \item Stores the provided `report' in the database associated with the
        specified `patientID'.
    \end{itemize}
  \end{item}
  \begin{item}
    output:
    \begin{itemize}
      \item `success' is set to True if the storing operation is successful,
        False otherwise.
    \end{itemize}
  \end{item}
  \begin{item}
    exception:
    \begin{itemize}
      \item Throws `ReportStorageException' if there is an issue storing the
        report.
    \end{itemize}
  \end{item}
\end{itemize}

\noindent retrieveReport():
\begin{itemize}
  \begin{item}
    transition:
    \begin{itemize}
      \item Retrieves the radiology report associated with the specified
        `patientID' from the database.
    \end{itemize}
  \end{item}
  \begin{item}
    output:
    \begin{itemize}
      \item `report' contains the retrieved radiology report.
    \end{itemize}
  \end{item}
  \begin{item}
    exception:
    \begin{itemize}
      \item Throws `ReportRetrievalException' if there is an issue retrieving the report.
    \end{itemize}
  \end{item}
\end{itemize}

\subsubsection{Local Functions}
N/A

\newpage

\section{MIS of User Authentication/Management Module} \label{mUserAuthMgmt}

\subsection{Module}
UserAuthMgmt

\subsection{Uses}
N/A

\subsection{Syntax}

\subsubsection{Exported Constants}
N/A

\subsubsection{Exported Access Programs}

\begin{center}
  \begin{tabular}{p{4cm} p{5cm} p{3cm} p{5cm}}
    \hline
    \textbf{Name} & \textbf{In} & \textbf{Out} & \textbf{Exceptions} \\
    \hline
    authenticateUser & username: str, password: str & status: bool &
      InvalidCredentialsException, UserNotFoundException \\
    createUserAccount & username: str, password: str & success: bool &
      UserCreationException \\
    deleteUserAccount & username: str, password: str & success: bool &
      UserDeletionException \\
    checkAuthentication & username: str & isAuthorized: bool & - \\
    \hline
  \end{tabular}
\end{center}

\subsection{Semantics}

\subsubsection{State Variables}
N/A

\subsubsection{Environment Variables}
N/A

\subsubsection{Assumptions}
N/A

\subsubsection{Access Routine Semantics}

\noindent authenticateUser():
\begin{itemize}
  \begin{item}
    transition:
    \begin{itemize}
      \item Verifies the provided `username' and `password' for authentication.
    \end{itemize}
  \end{item}
  \begin{item}
    output:
    \begin{itemize}
      \item `status' is set to True if authentication is successful, False
        otherwise.
    \end{itemize}
  \end{item}
  \begin{item}
    exception:
    \begin{itemize}
      \item Throws `InvalidCredentialsException' if the provided credentials
        are invalid.
      \item Throws `UserNotFoundException' if the specified user is not found.
    \end{itemize}
  \end{item}
\end{itemize}

\noindent createUserAccount():
\begin{itemize}
  \begin{item}
    transition:
    \begin{itemize}
      \item Creates a user account with the provided `username' and `password'.
    \end{itemize}
  \end{item}
  \begin{item}
    output:
    \begin{itemize}
      \item `success' is set to True if the account creation is successful,
        False otherwise.
    \end{itemize}
  \end{item}
  \begin{item}
    exception:
    \begin{itemize}
      \item Throws `UserCreationException' if there is an issue creating the
        user account.
    \end{itemize}
  \end{item}
\end{itemize}

\noindent deleteUserAccount():
  \begin{itemize}
    \begin{item}
      transition:
      \begin{itemize}
        \item Deletes the user account associated with the specified `username'.
      \end{itemize}
    \end{item}
    \begin{item}
      output:
      \begin{itemize}
        \item `success' is set to True if the account deletion is successful,
          False otherwise.
      \end{itemize}
    \end{item}
    \begin{item}
      exception:
      \begin{itemize}
        \item Throws `UserDeletionException' if there is an issue deleting the
          user account.
      \end{itemize}
    \end{item}
\end{itemize}

\noindent checkAuthentication():
\begin{itemize}
  \begin{item}
    transition:
    \begin{itemize}
      \item Checks whether the specified `username' is currently authorized.
    \end{itemize}
  \end{item}
  \begin{item}
    output:
    \begin{itemize}
      \item `isAuthorized' is set to True if the user is authorized, False
        otherwise.
    \end{itemize}
  \end{item}
  \begin{item}
    exception: N/A
  \end{item}
\end{itemize}

\subsubsection{Local Functions}
N/A

\newpage

\section{MIS of App GUI Module} \label{mAppGUI}

\subsection{Module}
AppGUI

\subsection{Uses}

\begin{itemize}
  \item Login
  \item PerfScan
  \item ViewResults
\end{itemize}

\subsection{Syntax}

\subsubsection{Exported Constants}
N/A

\subsubsection{Exported Access Programs}

\begin{center}
  \begin{tabular}{p{4cm} p{3cm} p{3cm} p{3cm}}
    \hline
    \textbf{Name} & \textbf{In} & \textbf{Out} & \textbf{Exceptions} \\
    \hline
    displayLoginPage & - & - & - \\
    displayScanPage & - & - & - \\
    displayResultsPage & - & - & - \\
    \hline
  \end{tabular}
\end{center}

\subsection{Semantics}

\subsubsection{State Variables}
N/A

\subsubsection{Environment Variables}
N/A

\subsubsection{Assumptions}
N/A

\subsubsection{Access Routine Semantics}

\noindent displayLoginPage():
\begin{itemize}
  \item transition: Navigates to and displays the login page for the
    application.
  \item output: N/A 
  \item exception: N/A
\end{itemize}

\noindent displayScanPage():
\begin{itemize}
  \item transition: Navigates to and displays the page for inputting an x-ray
    image for scanning.
  \item output: N/A
  \item exception: N/A
\end{itemize}

\noindent displayResultsPage():
\begin{itemize}
  \item transition: Navigates to and displays the page for viewing scan
    results and reports.
  \item output: N/A 
  \item exception: N/A
\end{itemize}

\subsubsection{Local Functions}
N/A

\newpage

\section{MIS of Login Module} \label{Module} 
\subsection{Module}
Login
\subsection{Uses}
N/A
\subsection{Syntax}

\subsubsection{Exported Constants}
N/A
\subsubsection{Exported Access Programs}

\begin{center}
\begin{tabular}{p{2cm} p{5cm} p{4cm} p{5cm}}
\hline
\textbf{Name} & \textbf{In} & \textbf{Out} & \textbf{Exceptions} \\
\hline
login & username: str, password: str & loginStatus: bool & InvalidCredentialsException, UserNotFoundException \\
\hline
\end{tabular}
\end{center}

\subsection{Semantics}

\subsubsection{State Variables}
N/A
\subsubsection{Environment Variables}
N/A
\subsubsection{Assumptions}
N/A
\subsubsection{Access Routine Semantics}

\noindent login():
\begin{itemize}
\item transition: \begin{itemize}
    \item Authenticates the provided 'username' and 'password'.
\end{itemize}
\item output: \begin{itemize}
    \item 'loginStatus' is set to True if login is successful, False otherwise.
\end{itemize}
\item exception: \begin{itemize}
    \item Throws 'InvalidCredentialsException' if the provided credentials are invalid.
    \item Throws 'UserNotFoundException' if the specified user is not found.
\end{itemize}
\end{itemize}

\subsubsection{Local Functions}
N/A
\newpage

\section{MIS of Perform Scan Module} \label{Module} 
\subsection{Module}
PerfScan
\subsection{Uses}
N/A
\subsection{Syntax}

\subsubsection{Exported Constants}
N/A
\subsubsection{Exported Access Programs}

\begin{center}
\begin{tabular}{p{3cm} p{4cm} p{2cm} p{5cm}}
\hline
\textbf{Name} & \textbf{In} & \textbf{Out} & \textbf{Exceptions} \\
\hline
initiateScan & img: X-Ray Image & - & InvalidImageFormatException \\
\hline
\end{tabular}
\end{center}

\subsection{Semantics}

\subsubsection{State Variables}
N/A
\subsubsection{Environment Variables}
N/A
\subsubsection{Assumptions}
N/A
\subsubsection{Access Routine Semantics}

\noindent initiateScan():
\begin{itemize}
\item transition: \begin{itemize}
    \item Receives the input 'img' from the user to initiate the scanning process.
\end{itemize}
\item output: N/A 
\item exception: \begin{itemize}
    \item Throws 'InvalidImageFormatException' if a non-JPEG/JPG image is used as input.
\end{itemize}
\end{itemize}

\subsubsection{Local Functions}
N/A
\newpage

\section{MIS of View Results Module} \label{Module} 
\subsection{Module}
ViewResults
\subsection{Uses}
N/A
\subsection{Syntax}

\subsubsection{Exported Constants}
N/A
\subsubsection{Exported Access Programs}

\begin{center}
\begin{tabular}{p{3cm} p{5cm} p{2cm} p{2cm}}
\hline
\textbf{Name} & \textbf{In} & \textbf{Out} & \textbf{Exceptions} \\
\hline
displayReport & report: Radiology Report & - & - \\
\hline
\end{tabular}
\end{center}

\subsection{Semantics}

\subsubsection{State Variables}
N/A
\subsubsection{Environment Variables}
N/A
\subsubsection{Assumptions}
N/A
\subsubsection{Access Routine Semantics}

\noindent displayReport():
\begin{itemize}
\item transition: \begin{itemize}
    \item Displays the generated radiology report on the GUI.
\end{itemize}
\item output: N/A
\item exception: N/A
\end{itemize}

\subsubsection{Local Functions}
N/A
\newpage

\section{MIS of AI Model Module} \label{Module} 
\subsection{Module}
AIModel
\subsection{Uses}
\begin{itemize}
    \item ChestXRayRead
    \item ResultGen
\end{itemize}
\subsection{Syntax}

\subsubsection{Exported Constants}
N/A
\subsubsection{Exported Access Programs}

\begin{center}
\begin{tabular}{p{3cm} p{4cm} p{4cm} p{5cm}}
\hline
\textbf{Name} & \textbf{In} & \textbf{Out} & \textbf{Exceptions} \\
\hline
processImg & img: JPEG image & procImg: Processed Image & InvalidImageFormatException \\
generateResults & procImg: Processed Image & classification: Disease Classification & - \\
\hline
\end{tabular}
\end{center}

\subsection{Semantics}

\subsubsection{State Variables}
N/A
\subsubsection{Environment Variables}
N/A
\subsubsection{Assumptions}
N/A
\subsubsection{Access Routine Semantics}

\noindent processImg():
\begin{itemize}
\item transition: \begin{itemize}
    \item Uses the trained model to process the given 'img'.
\end{itemize}
\item output: \begin{itemize}
    \item 'procImg' contains the processed information and findings from the chest X-ray analysis.
\end{itemize}
\item exception: \begin{itemize}
    \item Throws ’InvalidImageFormatException’ if the provided ’image’ is not a JPEG or JPG image.
\end{itemize} 
\end{itemize}

\noindent generateResults():
\begin{itemize}
\item transition: \begin{itemize}
    \item Uses the trained model to interpret the processed image and generate a disease classification.
\end{itemize}
\item output: \begin{itemize}
    \item 'classification' contains the generated disease classification for each disease.
\end{itemize}
\item exception: N/A
\end{itemize}

\subsubsection{Local Functions}
N/A
\newpage

\section{MIS of NLP Model Module} \label{Module} 
\subsection{Module}
NLPModel
\subsection{Uses}
\begin{itemize}
    \item RepCompGen
\end{itemize}
\subsection{Syntax}

\subsubsection{Exported Constants}
N/A
\subsubsection{Exported Access Programs}

\begin{center}
\begin{tabular}{p{3cm} p{4cm} p{4cm} p{5cm}}
\hline
\textbf{Name} & \textbf{In} & \textbf{Out} & \textbf{Exceptions} \\
\hline
generateReport & report: Radiology Report & nlp: NLP Report & InvalidReportFormatException \\
\hline
\end{tabular}
\end{center}

\subsection{Semantics}

\subsubsection{State Variables}
N/A
\subsubsection{Environment Variables}
N/A
\subsubsection{Assumptions}
N/A
\subsubsection{Access Routine Semantics}

\noindent generateReport():
\begin{itemize}
\item transition: \begin{itemize}
    \item Uses the RepCompGen module to generate a radiology report.
\end{itemize}
\item output: \begin{itemize}
    \item 'nlp' contains the generated NLP report.
\end{itemize}
\item exception: \begin{itemize}
    \item Throws 'InvalidReportFormatException' if the provided 'report' is in an invalid format.
\end{itemize}
\end{itemize}

\subsubsection{Local Functions}
N/A
\newpage

\section{MIS of Backend Module} \label{Module} 
\subsection{Module}
Backend
\subsection{Uses}
\begin{itemize}
    \item UserAuthMgmt
    \item MedInstInter
    \item DatabaseOps
\end{itemize}
\subsection{Syntax}

\subsubsection{Exported Constants}
N/A
\subsubsection{Exported Access Programs}

\begin{center}
\begin{tabular}{p{3cm} p{4cm} p{4cm} p{5cm}}
\hline
\textbf{Name} & \textbf{In} & \textbf{Out} & \textbf{Exceptions} \\
\hline
connectDatabase & credentials: str & connectionStatus: bool & InvalidCredentialsException \\
disconnectDatabase & - & success: bool & - \\
\hline
\end{tabular}
\end{center}

\subsection{Semantics}

\subsubsection{State Variables}

\subsubsection{Environment Variables}

\subsubsection{Assumptions}

\subsubsection{Access Routine Semantics}

\noindent connectDatabase():
\begin{itemize}
\item transition: N/A 
\item output: \begin{itemize}
    \item 'connectionStatus' is set to True if the connection is successful, False otherwise.
\end{itemize}
\item exception: \begin{itemize}
    \item Throws 'InvalidCredentialsException' if the provided credentials are invalid.
\end{itemize}
\end{itemize}

\noindent disconnectDatabase():
\begin{itemize}
\item transition: N/A 
\item output: \begin{itemize}
    \item 'success' is set to True if the disconnection is successful, False otherwise.
\end{itemize} 
\item exception: N/A 
\end{itemize}

\subsubsection{Local Functions}
N/A
\newpage

\section{MIS of App Controller Module} \label{Module} 
\subsection{Module}
AppController
\subsection{Uses}
\begin{itemize}
    \item AIModel
    \item NLPModel
    \item AppGUI
    \item Backend
\end{itemize}
\subsection{Syntax}

\subsubsection{Exported Constants}
N/A
\subsubsection{Exported Access Programs}

\begin{center}
\begin{tabular}{p{3cm} p{4cm} p{4cm} p{3cm}}
\hline
\textbf{Name} & \textbf{In} & \textbf{Out} & \textbf{Exceptions} \\
\hline
accessBackend & - & - & - \\
accessGUI & - & - & - \\
accessAI & - & - & - \\
accessNLP & - & - & - \\

\hline
\end{tabular}
\end{center}

\subsection{Semantics}

\subsubsection{State Variables}
N/A
\subsubsection{Environment Variables}
N/A
\subsubsection{Assumptions}
N/A
\subsubsection{Access Routine Semantics}

\noindent accessBackend():
\begin{itemize}
\item transition: \begin{itemize}
    \item Controller accesses the backend server.
\end{itemize}
\item output: N/A
\item exception: N/A
\end{itemize}

\noindent accessGUI():
\begin{itemize}
\item transition: \begin{itemize}
    \item Controller accesses the application GUI.
\end{itemize} 
\item output: N/A 
\item exception: N/A
\end{itemize}

\noindent accessAI():
\begin{itemize}
\item transition: \begin{itemize}
    \item Controller accesses the AI Model.
\end{itemize}
\item output: N/A
\item exception: N/A
\end{itemize}

\noindent accessNLP():
\begin{itemize}
\item transition: \begin{itemize}
    \item Controller acceses the NLP Model.
\end{itemize}
\item output: N/A 
\item exception: N/A
\end{itemize}

\subsubsection{Local Functions}
N/A

\newpage

\bibliographystyle {plainnat}
\bibliography {../../../refs/References}

\newpage

\section{Appendix} \label{Appendix}

\end{document}