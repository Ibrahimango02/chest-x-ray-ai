% THIS DOCUMENT IS FOLLOWS THE VOLERE TEMPLATE BY Suzanne Robertson and James Robertson
% ONLY THE SECTION HEADINGS ARE PROVIDED
%
% Initial draft from https://github.com/Dieblich/volere
%
% Risks are removed because they are covered by the Hazard Analysis
\documentclass[12pt]{article}

\usepackage{float}
\usepackage{booktabs}
\usepackage{tabularx}
\usepackage{hyperref}
\usepackage{blindtext}
\usepackage[english]{babel}
\hypersetup{
    bookmarks=true,         % show bookmarks bar?
      colorlinks=true,      % false: boxed links; true: colored links
    linkcolor=red,          % color of internal links (change box color with linkbordercolor)
    citecolor=green,        % color of links to bibliography
    filecolor=magenta,      % color of file links
    urlcolor=cyan           % color of external links
}

\newcommand{\lips}{\textit{Insert your content here.}}

%%% Comments

\usepackage{color}

\newif\ifcomments\commentstrue %displays comments
%\newif\ifcomments\commentsfalse %so that comments do not display

\ifcomments
\newcommand{\authornote}[3]{\textcolor{#1}{[#3 ---#2]}}
\newcommand{\todo}[1]{\textcolor{red}{[TODO: #1]}}
\else
\newcommand{\authornote}[3]{}
\newcommand{\todo}[1]{}
\fi

\newcommand{\wss}[1]{\authornote{blue}{SS}{#1}} 
\newcommand{\plt}[1]{\authornote{magenta}{TPLT}{#1}} %For explanation of the template
\newcommand{\an}[1]{\authornote{cyan}{Author}{#1}}

%% Common Parts

\newcommand{\progname}{ProgName} % PUT YOUR PROGRAM NAME HERE
\newcommand{\authname}{Team \#, Team Name
\\ Student 1 name
\\ Student 2 name
\\ Student 3 name
\\ Student 4 name} % AUTHOR NAMES                  

\usepackage{hyperref}
    \hypersetup{colorlinks=true, linkcolor=blue, citecolor=blue, filecolor=blue,
                urlcolor=blue, unicode=false}
    \urlstyle{same}
                                


\begin{document}

\title{Software Requirements Specification for \progname: AI for Chest X-Ray Read}
\author{\authname}
\date{\today}
  
\maketitle

~\newpage

\pagenumbering{roman}

\tableofcontents

~\newpage

\section*{Revision History}

\begin{tabularx}{\textwidth}{p{4cm}p{4cm}X}
\toprule {\textbf{Date}} & {\textbf{Developer(s)}} & {\textbf{Notes}}\\
\midrule
October 1st 2023 & Allison Cook, Ibrahim Issa, Mohaansh Pranjal, Nathaniel Hu, Tushar Aggarwal & Initial Draft of SRS document \\
October 4th 2023 & Name & Notes\\
October 6th 2023 & Allison Cook, Ibrahim Issa, Mohaansh Pranjal, Nathaniel Hu, Tushar Aggarwal  & Edits to SRS document \\
\bottomrule
\end{tabularx}

~\\

~\newpage
\section{Purpose of the Project}
This project aims to provide a system that will reduce the amount of time radiologists and medical professionals will need to spend reviewing chest x-rays of patients. Users of the completed solution proposed in this project will be able to upload a chest x-ray and receive a generated diagnostic radiology report outlining any abnormalities in the given image.

\subsection{User Business}
The user businesses are medical institutions (e.g. hospitals) and other medical imaging businesses (e.g. diagnostic centres). These businesses aim to review and analyze patient chest x-rays, perform diagnostics, report their findings and present the results to their patients and/or relevant third parties as soon as possible. 

\subsection{Goals of the Project}
The primary goal of the project is to provide accurate detection of abnormality given an x-ray image and generate a diagnostic radiology report outlining any abnormalities. The project's other goals are to provide secure and remote access to the system while having an intuitive user interface. A further stretch goal of the project is to be able to generate structured or free-formed radiology diagnostic reports using natural language processing (NLP).

\section{Stakeholders}
This section outlines the various stakeholders of this project's proposed solution, and describes them each in detail with regards to their relevance to this project's success. These include the clients, customers, other stakeholders and hands-on users. Their personas, assigned priorities, user participation and the maintenance users and service technicians are also described in detail.

\subsection{Client}
The following are the clients of project's proposed solution:
\begin{itemize}
    \item \textbf{Doctors}: want the tedious work of analyzing chest x-rays, performing diagnostics and writing radiology reports to be semi-/fully-automated
    \item \textbf{Patients}: wants chest x-ray results faster with the same accuracy
    \item \textbf{Diagnostics Teams}: wants to maximize the number of chest x-rays performed and processed for patients
\end{itemize}
These are the people expected to be the primary users of the project's proposed solution. They are expected to use this proposed solution primarily in their daily work as medical professionals, or benefit from it (i.e. the patients).

\subsection{Customer}
The customers of the project's proposed solution are described as follows:
\begin{itemize}
    \item \textbf{Medical Institutions}: want to process chest x-rays faster to get results for patients in time-critical situations (e.g. detect time-sensitive diseases)
    \item \textbf{Diagnostic Centres}: want to process chest x-rays faster to maximize number of chest x-rays processed for patients, but not necessarily as time-sensitive (e.g. routine checkups)
\end{itemize}

\subsection{Other Stakeholders}
The following are the other stakeholders of the project’s proposed solution: 
\begin{itemize}
    \item Medical institutions’ IT departments 
    \item Developers 
\end{itemize}
These are the other stakeholders' who do not directly use the project's proposed solution, but are involved in supporting it in carrying out its intended functions successfully.

\subsection{Hands-On Users of the Project}
The following are the hands-on users of the project’s proposed solution:
\begin{itemize}
    \item \textbf{Medical professionals} using this software to perform initial analysis of chest x-rays and diagnose diseases and other health conditions 
\end{itemize}
Like as was mentioned earlier, these users are expected to use the project's proposed solution in their daily work lives to support their work.

\subsection{Personas}
The following are descriptions outlining the personas modelling the respective stakeholders and users of the project’s proposed solution:
\begin{itemize}
    \item \textbf{Doctors/Medical Professionals}
    \begin{itemize}
        \item Stressed, under pressure to analyze chest x-rays, perform diagnoses and present results to patients and all other relevant stakeholders as soon as possible. 
        \item Want the information presented to them such that they can glance over it quickly and grab the relevant information to present 
        \item May be looking to verify the position of lines and tubes in patients in the ICU or during/after interventions 
    \end{itemize}
    \item \textbf{Patients}
    \begin{itemize}
        \item May be stressed, worried about chest x-ray results for variety of reasons 
        \item May be waiting on results for diagnosis of possible cardiac or lung conditions 
        \item For ruling out diseases for regulatory reasons such as immigration or occupational health assessments 
    \end{itemize}
    \item \textbf{Medical Institution IT Department}
    \begin{itemize}
        \item Want to ensure security of systems to ensure patient privacy of their medical records 
    \end{itemize}
\end{itemize}

\subsection{Priorities Assigned to Users}
The following are the assigned priorities of the users of this project's proposed solution:
\begin{itemize}
    \item \textbf{Doctor/Medical Professional}: key user, the expected primary user of the proposed solution to assist them in their work
    \item \textbf{Patient}: secondary user, benefits from having radiology analyses and diagnostic reports generated from their chest x-rays with faster results
    \item \textbf{Medical Institution IT Department}: tertiary user, supports the function of the proposed solution
\end{itemize}

\subsection{User Participation}
The user participation of the main users of the project’s proposed solution are described in further detail below:
\begin{itemize}
    \item \textbf{Doctors/Medical Professionals}
    \begin{itemize}
        \item User feeds chest x-ray images into the software for analysis and diagnosis
        \item User reviews output diagnostic report produced by the software 
    \end{itemize} 
    \item \textbf{Patients}
    \begin{itemize}
        \item User is presented the results of the diagnostic report by the doctor/medical professional
    \end{itemize}
    \item \textbf{Medical Institution IT Department}
    \begin{itemize}
        \item User authorizes the chest X-ray images to taken from the medical institution’s medical information systems by authorized users
    \end{itemize}
\end{itemize}

\subsection{Maintenance Users and Service Technicians}
The maintenance users and service technicians of the project’s proposed solution are described below:
\begin{itemize}
    \item \textbf{Developers/Testers}, those responsible for the development of the project’s proposed solution, as well as maintaining it (fixing bugs, adding updates implementing new features/improving existing ones) 
\end{itemize}

\section{Mandated Constraints}
This section describes the various constraints placed on this project's proposed solution in more detail. These include the solution constraints, the implementation environment of the current system, supporting partner or collaborative applications and existing off-the-shelf software. The anticipated workplace environment and schedule, budget, and enterprise constraints are also described in more detail here.

\subsection{Solution Constraints}
The following are constraints placed on the project’s proposed solution. Each is described in more detail, including its rationale and fit criteria, as shown below:
\begin{itemize}
    \item The product shall operate as a web application  ---- \textbf{This is a possible answer or something similar}
    \begin{description} 
        \item[Rationale:] This will permit the users at different hospitals to use the system without any change to their institutional system
        \item[Fit Criterion:] The product will contain 
    \end{description}
\end{itemize}

\subsection{Implementation Environment of the Current System}
The following are constraints resulting from the implementation environment of the current system:

\begin{itemize}
    \item Model source code is 
    \item \textbf{NAME!} library used for model training
    \item 
\end{itemize}

\subsection{Partner or Collaborative Applications}
The following are partner or collaborative applications that the project’s proposed solution is expected to work with:

\begin{itemize}
    \item Medical institution’s internal IT systems and databases where the patients’ chest X-rays and other relevant medical records are stored
\end{itemize}

\subsection{Off-the-Shelf Software}
The following are off-the-shelf software with functionality comparable to the project’s proposed solution: 
\begin{itemize}
    \item ??? Name ???
\end{itemize}

\subsection{Anticipated Workplace Environment}
The anticipated workplace environment for the project’s proposed solution is described as follows:
\begin{itemize}
    \item \textbf{Medical Institutions}: chest x-rays taken to diagnose diseases or conditions, verify positions of lines and tubes in ICU patients before/after interventions; more time-critical
    \item \textbf{Diagnostics Offices}: chest x-rays taken to diagnose diseases or conditions, for routine checkups or for regulatory reasons (e.g. immigration, occupational health assessments); less time-critical
\end{itemize}


\subsection{Schedule Constraints}
The schedule constraints (i.e. project deadlines) for the project’s proposed solution are described as follows:
\begin{itemize}
    \item \textbf{Proof of Concept Demonstration}: November 13 - 24, 2023, demonstrate that key parts of the proposed solution are viable
    \item \textbf{Revision 0 Demonstration}: February 5 - 16, 2024, demonstrate the functionality of the initial version of the proposed solution
    \item \textbf{Final/Revision 1 Demonstration}: March 18 - 24, 2024, demonstrate the functionality of the first revision of the proposed solution
\end{itemize}

\subsection{Budget Constraints}
There is a budget constraint of \$750 for this project's proposed solution. However, no necessary funding is needed for the project and no additional pieces of hardware or software will be purchased for this project. 

\subsection{Enterprise Constraints}
Given that the project’s proposed solution is expected to interface with a medical institution’s IT systems, the following enterprise constraints apply:
\begin{itemize}
    \item \textbf{Security}
    \begin{itemize}
        \item The system handles individual patients health information and must only be accessed by authorized individuals and protected from unauthorized access
    \end{itemize}
    \item \textbf{Computation Utilization}
    \begin{itemize}
        \item The system should not overload the medical institution's IT systems needed to support its proper functionality
    \end{itemize}
\end{itemize}

\section{Naming Conventions and Terminology}
This section describes all of the naming conventions and terminology relevant to documenting this project's proposed solution. This mainly includes the glossary of all terms (including acronyms) that are used by stakeholders involved in the project.

\subsection{Glossary of All Terms, Including Acronyms, Used by Stakeholders
involved in the Project}
\textit{N/A}: there are no naming conventions and terminology identified as relevant to documenting this project's proposed solution that are used by stakeholders involved in this project.

\section{Relevant Facts And Assumptions}
This section includes all of the relevant facts, business rules and assumptions relevant to this project's proposed solution, described in further detail below.

\subsection{Relevant Facts}
The following are facts relevant to the project's proposed solution:
\begin{itemize}
    \item Chest x-rays are the most common medical imaging modality 
    \item Chest x-rays constitute 40\% of the 3.6 billion medical imaging procedures performed worldwide each year 
\end{itemize}

\subsection{Business Rules}
The business rules relevant to the project's proposed solution are described in detail below:
\begin{itemize}
    \item proper security protocols are followed when retrieving and storing patients' medical records and data (i.e. chest x-rays)
    \item proper patient privacy policies are followed when processing and sharing patients' medical information with other parties (i.e. only shared with authorized parties)
    \item proper patient data protection policies are followed when processing and storing patients' medical information
\end{itemize}

\subsection{Assumptions}
The following are assumptions made about the project's proposed solution:
\begin{itemize}
    \item The system has access to a DICOM server with the required chest X-ray images 
    \item The accuracy of the system will not be 100\%
    \item The model will be trained with a smaller section of the chest X-ray library due to limited computational power
\end{itemize} 

\section{The Scope of the Work}
This section describes the scope of the work to be done for the project's proposed solution. This includes the current situation, the context of the work, work partitioning and a business use case scenario.

\subsection{The Current Situation}
Chest x-rays, constituting 40\% of the 3.6 billion annual medical imaging procedures globally, serve as a primary diagnostic tool for various lung and heart conditions.

Radiologists and healthcare professionals face significant time constraints in analyzing chest X-rays, potentially leading to critical delays with life-threatening implications for patients. 

\subsection{The Context of the Work}
To develop a comprehensive solution addressing the time-intensive nature of chest X-ray analysis for healthcare professionals. 
\textbf{Context diagram??}

\subsection{Work Partitioning}
The following details how the work will be partitioned for this project:
\begin{figure}[h]
    \caption{Table 1: Work Partitioning}
    \begin{tabularx}{\textwidth}{|X|X|X|}
    \hline
    Event Name & Input/Output & Summary \\
    \hline
    input chest x-ray image & chest x-ray image (in), disease/abnormality detection (out) & user inputs chest x-ray image, gets disease/abnormality detection \\
    \hline
    input chest x-ray image & chest x-ray image (in), diagnostic radiology report information (out) & user inputs chest x-ray image, gets diagnostic radiology report information \\ 
    \hline
    \end{tabularx}
\end{figure}

\begin{itemize}
    \item Implementation of a Computer Vision and Neural Network component for automated detection of abnormalities in chest X-rays.
    \item Development of a User interface component to facilitate user input and display diagnostic reports.
    \item Integration of a Security component to safeguard sensitive user information. 
\end{itemize}

\subsection{Specifying a Business Use Case (BUC)}
Business Use Case Scenario: 
\begin{enumerate}
    \item \textbf{Patient arrival:} A patient arrives at the emergency room with severe respiratory symptoms, the attending doctor orders a chest X-ray for assessment of the patient’s lungs. 
    \item \textbf{AI chest X-ray:} An X-ray of the patient’s chest is taken and input into the system. The system identifies key abnormalities in the patient. 
    \item \textbf{Diagnostic report:} The system generates a list of the identified findings and their severity levels. The user interface presents this information in a clear and concise manner. 
    \item \textbf{Treatment:} The doctor quickly reviews the AI-generated report without examining the X-ray and begins treatment. 
\end{enumerate}  

\section{Business Data Model and Data Dictionary}
\subsection{Business Data Model}
\lips
\subsection{Data Dictionary}
\lips

\section{The Scope of the Product}
\subsection{Product Boundary}
The application encompasses the entire life cycle of the Automated Chest X-ray Diagnosis System, from the initial input of a chest X-ray image to the generation of a structured radiology report. It includes all components such as the Computer Vision and Neural Network modules, User Interface, and Security. 

\subsection{Product Use Case Table}

\begin{figure}[H]
    \caption{Table 2: Product Use Case Table}
    \begin{tabularx}{\textwidth}{|X|X|}
    \hline
    Use Case ID & Use Case Description \\
    \hline
    PUC-001 & Process chest X-ray image using computer vision module \\
    \hline
    PUC-002 & Generate a list of identified findings \\
    \hline
    PUC-003 & Convert findings into a structured diagnostic report \\
    \hline
    PUC-004 & Display diagnostic report on the user interface \\
    \hline
    \end{tabularx}
\end{figure}

\subsection{Individual Product Use Cases (PUC's)}
\begin{enumerate}
    \begin{item}
        PUC-001:
        \begin{itemize}
        \item Description: the system takes a chest X-ray image as input and processes it to identify abnormalities.
        \item Actors: computer vision module, chest x-ray image
        \item Preconditions: valid chest x-ray image input is provided
        \item Postconditions: processed image with identified abnormalities
        \end{itemize}
    \end{item}
    \begin{item}
        PUC-002:
        \begin{itemize}
            \item Description: the system generates a comprehensive list of identified findings based on the processed chest x-ray image
            \item Actors: computer vision module, neural network module 
            \item Preconditions: processed image with abnormalities is provided
            item Postconditions: list of identified findings is generated 
        \end{itemize}
    \end{item}
    \begin{item}
        PUC-003:
        \begin{itemize}
            \item Description: The system converts the list of identified findings into a structured radiology report 
            \item Actors: list of findings
            \item Preconditions: list of identified findings is provided
            \item Postconditions: diagnostic report is generated
        \end{itemize}
    \end{item}
    \begin{item}
        PUC-004:
        \begin{itemize}
            \item Description: The user interface module displays the diagnostic report for the user
            \item Actors: user interface module, user
            \item Preconditions: diagnostic report is provided
            \item Postconditions: diagnostic report is displayed on the user interface
        \end{itemize}
    \end{item}
\end{enumerate}

\section{Functional Requirements}
\subsection{Functional Requirements}
\textbf{Requirement \#:} 1 \hfill \textbf{Requirement type:} \hfill \textbf{Use case:}  \\
\textbf{Description:} The system shall accept and read jpeg images as input. \\
\textbf{Rationale: } \\
\textbf{Fit Criterion:}

\vspace{2mm}
\noindent
\textbf{Requirement \#:} 2 \hfill \textbf{Requirement type:} \hfill \textbf{Use case:} \\
\textbf{Description:} The system shall display jpeg images. \\
\textbf{Rationale: } \\
\textbf{Fit Criterion:} 

\vspace{2mm}
\noindent
\textbf{Requirement \#:} 3 \hfill \textbf{Requirement type:} \hfill \textbf{Use case:} \\
\textbf{Description:} The system shall generate and display a report outlining the findings.  \\
\textbf{Rationale: } \\
\textbf{Fit Criterion:}

\vspace{2mm}
\noindent
\textbf{Requirement \#:} 4 \hfill \textbf{Requirement type:} \hfill \textbf{Use case:} \\
\textbf{Description:} The system shall be able to fetch patients’ records on retrieval request by the user.  \\
\textbf{Rationale: } \\
\textbf{Fit Criterion:}

\vspace{2mm}
\noindent
\textbf{Requirement \#:} 5 \hfill \textbf{Requirement type:} \hfill \textbf{Use case:} \\
\textbf{Description:} The system shall accurately detect and classify abnormalities or normality in a given X-ray image.   \\
\textbf{Rationale: } \\
\textbf{Fit Criterion:}

\vspace{2mm}
\noindent
\textbf{Requirement \#:} 6 \hfill \textbf{Requirement type:} \hfill \textbf{Use case:} \\
\textbf{Description:} The system shall be accessible remotely via a web interface.   \\
\textbf{Rationale: } \\
\textbf{Fit Criterion:}

\vspace{2mm}
\noindent
\textbf{Requirement \#:} 7 \hfill \textbf{Requirement type:} \hfill \textbf{Use case:} \\
\textbf{Description:} The system shall convert images from a DICOM file to a jpeg, jpg or other suitable image format to be processed by the ML algorithm.  \\
\textbf{Rationale: } \\
\textbf{Fit Criterion:}

\section{Look and Feel Requirements}
\subsection{Appearance Requirements}
The 
\subsection{Style Requirements}
\lips

\section{Usability and Humanity Requirements}
\subsection{Ease of Use Requirements}
\lips
\subsection{Personalization and Internationalization Requirements}
\lips
\subsection{Learning Requirements}
\lips
\subsection{Understandability and Politeness Requirements}
\lips
\subsection{Accessibility Requirements}
\lips

\section{Performance Requirements}
\subsection{Speed and Latency Requirements}
\lips
\subsection{Safety-Critical Requirements}
\lips
\subsection{Precision or Accuracy Requirements}
\lips
\subsection{Robustness or Fault-Tolerance Requirements}
\lips
\subsection{Capacity Requirements}
\lips
\subsection{Scalability or Extensibility Requirements}
\lips
\subsection{Longevity Requirements}
\lips

\section{Operational and Environmental Requirements}
\subsection{Expected Physical Environment}
\lips
\subsection{Wider Environment Requirements}
\lips
\subsection{Requirements for Interfacing with Adjacent Systems}
\lips
\subsection{Productization Requirements}
\lips
\subsection{Release Requirements}
\lips

\section{Maintainability and Support Requirements}
\subsection{Maintenance Requirements}
\lips
\subsection{Supportability Requirements}
\lips
\subsection{Adaptability Requirements}
\lips

\section{Security Requirements}
\subsection{Access Requirements}
\lips
\subsection{Integrity Requirements}
\lips
\subsection{Privacy Requirements}
\lips
\subsection{Audit Requirements}
\lips
\subsection{Immunity Requirements}
\lips

\section{Cultural Requirements}
\subsection{Cultural Requirements}
\lips

\section{Compliance Requirements}
\subsection{Legal Requirements}
\lips
\subsection{Standards Compliance Requirements}
\lips

\section{Open Issues}
\lips

\section{Off-the-Shelf Solutions}
\subsection{Ready-Made Products}
\lips
\subsection{Reusable Components}
\lips
\subsection{Products That Can Be Copied}
\lips

\section{New Problems}
\subsection{Effects on the Current Environment}
\lips
\subsection{Effects on the Installed Systems}
\lips
\subsection{Potential User Problems}
\lips
\subsection{Limitations in the Anticipated Implementation Environment That May
Inhibit the New Product}
\lips
\subsection{Follow-Up Problems}
\lips

\section{Tasks}
\subsection{Project Planning}
\lips
\subsection{Planning of the Development Phases}
\lips

\section{Migration to the New Product}
\subsection{Requirements for Migration to the New Product}
\lips
\subsection{Data That Has to be Modified or Translated for the New System}
\lips

\section{Costs}
\lips
\section{User Documentation and Training}
\subsection{User Documentation Requirements}
\lips
\subsection{Training Requirements}
\lips

\section{Waiting Room}
\lips

\section{Ideas for Solution}
\lips

\newpage{}
\section*{Appendix --- Reflection}

The information in this section will be used to evaluate the team members on the
graduate attribute of Lifelong Learning.  Please answer the following questions:

\begin{enumerate}
  \item What knowledge and skills will the team collectively need to acquire to
  successfully complete this capstone project?  Examples of possible knowledge
  to acquire include domain specific knowledge from the domain of your
  application, or software engineering knowledge, mechatronics knowledge or
  computer science knowledge.  Skills may be related to technology, or writing,
  or presentation, or team management, etc.  You should look to identify at
  least one item for each team member.
  \item For each of the knowledge areas and skills identified in the previous
  question, what are at least two approaches to acquiring the knowledge or
  mastering the skill?  Of the identified approaches, which will each team
  member pursue, and why did they make this choice?
\end{enumerate}

\end{document}